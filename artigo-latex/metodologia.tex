\section{Metodologia} 

\begin{itemize}
  \item Identifica��o e escolha das versoes das ferramentas utilizadas
  \begin{itemize}
    \item Definir crit�rios de sele��o
    \item Selecionar ferramentas
  \end{itemize}
  \item Testes iniciais
  \begin{itemize}
    \item Execu��o a ferramenta sobre o Juliet
    \item Identifica��o de erros nos Makefiles do Juliet
    \item Corre��o dos erros encontrados
    \item Reexecu��o da ferramenta sobre o Juiet
  \end{itemize}
  \item Selecao dos parametros a serem extra�dos dos testes
  \begin{itemize}
    \item Parametros globais
    \begin{itemize}
      \item Total de arquivos analisados
      \item Total de arquivos n�o reportados
      \item Total de OKs (vulnerabilidade esperada foi encontrada)
      \item Total de fails (vulnerabilidade esperada n�o foi encontrada)
    \end{itemize}
    \item Parametros por arquivo
    \begin{itemize}
      \item Nome do arquivo
      \item Vulnerabilidade esperada
      \item Vulnerabilidade(s) encontrada(s)
      \item Vulnerabilidade foi encontrada? (OK ou fail)
    \end{itemize}
  \end{itemize}
  \item Automa��o do processo de an�lise do aquivo com o Clang
  \item Geracao de relatorio com as metricas de porcentagem de sucesso
\end{itemize}
