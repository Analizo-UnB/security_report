\section{Discuss�o}

\begin{itemize}
\item Considera��o sobre comportamento e peculiaridades do Clang
\item Pressuposto uma vulnerabilidade por arquivo
\item Percentual baixo, por�m comum
\item Maioria dos arquivos n�o obtiveram report
\end{itemize}

As vulnerabilidades e suas nomenclaturas identificadas pela ferramenta \"scan-build\" (presente na ferramenta Clang) 
n�o est�o alinhadas com as determinadas pelo NIST, sendo necess�ria, para este trabalho, uma aproxima��o. Foi levantada 
a hipot�se de exist�ncia de apenas uma vulnerabilidade por arquivo de teste. Deste modo, h� a possibilidade de que
a quantidade de ocorr�ncias de vulnerabilidades esperadas n�o corresponda � real. Entretanto, o modo de abordagem utilizado
diminui o impacto dessa poss�vel discrep�ncia.

O percentual de vulnerabilidades encontradas equivalentes as esperadas foi relativamente baixo, contudo, n�o foi
uma surpresa, j� que a grande maioria das ferramentas possuem tal desempenho. O fato de n�o ter encontrado nenhuma 
vulnerabilidade na grande maioria dos casos de teste influenciou bastante para o baixo percentual obtido, n�o estando, este fato, 
relacionado a uma interpreta��o err�nea, realmente apresentando um mal desempenho da ferramenta. No contexto dos casos de testes
que foram identificadas vulnerabilidades, pode ou n�o ter inconsist�ncias, dependendo da acur�cia da 
hip�tese levantada. No espa�o amostral das arquivos que foram encontradas vulnerabilidades, cerca de um quinto das vulnerabilidades 
encontradas foram equivalentes as esperadas, o que pode-se dizer satisfat�rio.
