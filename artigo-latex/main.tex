\documentclass[12pt]{article}

\usepackage{main}

\usepackage{graphicx,url}

%\usepackage[brazil]{babel}   
\usepackage[latin1]{inputenc}  

     
\sloppy

\title{Clang -- Juliet}

\author{Alexandre A. Barbosa\inst{1}, Edson Alves\inst{1}, Lucas Kanashiro\inst{1}, Paulo R. M. Meirelles\inst{1} }


\address{Faculdade UnB Gama (FGA) -- Universidade de Bras�lia (UnB)\\
  CEP: 72 444 - 240 -- Gama -- DF -- Brazil
  \email{\{alexandrealmeidabarbosa, kanashiro.duarte\}@gmail.com, \newline edsonalves@unb.br, paulo@softwarelivre.org}
}

\begin{document} 

\maketitle

\begin{abstract}
  This meta-paper describes the style to be used in articles and short papers
  for SBC conferences. For papers in English, you should add just an abstract
  while for the papers in Portuguese, we also ask for an abstract in
  Portuguese (``resumo''). In both cases, abstracts should not have more than
  10 lines and must be in the first page of the paper.
\end{abstract}
     
\begin{resumo} 
  Este meta-artigo descreve o estilo a ser usado na confec��o de artigos e
  resumos de artigos para publica��o nos anais das confer�ncias organizadas
  pela SBC. � solicitada a escrita de resumo e abstract apenas para os artigos
  escritos em portugu�s. Artigos em ingl�s dever�o apresentar apenas abstract.
  Nos dois casos, o autor deve tomar cuidado para que o resumo (e o abstract)
  n�o ultrapassem 10 linhas cada, sendo que ambos devem estar na primeira
  p�gina do artigo.
\end{resumo}

\section{Introdu��o}

Introdu��o.



\section{Metodologia} 

\begin{itemize}
  \item Identifica��o e escolha das versoes das ferramentas utilizadas
  \begin{itemize}
    \item Definir crit�rios de sele��o
    \item Selecionar ferramentas
  \end{itemize}
  \item Testes iniciais
  \begin{itemize}
    \item Execu��o a ferramenta sobre o Juliet
    \item Identifica��o de erros nos Makefiles do Juliet
    \item Corre��o dos erros encontrados
    \item Reexecu��o da ferramenta sobre o Juiet
  \end{itemize}
  \item Selecao dos parametros a serem extra�dos dos testes
  \begin{itemize}
    \item Parametros globais
    \begin{itemize}
      \item Total de arquivos analisados
      \item Total de arquivos n�o reportados
      \item Total de OKs (vulnerabilidade esperada foi encontrada)
      \item Total de fails (vulnerabilidade esperada n�o foi encontrada)
    \end{itemize}
    \item Parametros por arquivo
    \begin{itemize}
      \item Nome do arquivo
      \item Vulnerabilidade esperada
      \item Vulnerabilidade(s) encontrada(s)
      \item Vulnerabilidade foi encontrada? (OK ou fail)
    \end{itemize}
  \end{itemize}
  \item Automa��o do processo de an�lise do aquivo com o Clang
  \item Geracao de relatorio com as metricas de porcentagem de sucesso
\end{itemize}


\section{Resultados}

Resultados.



\section{Discuss�o}

Section titles must be in boldface, 13pt, flush left. There should be an extra
12 pt of space before each title. Section numbering is optional. The first
paragraph of each section should not be indented, while the first lines of
subsequent paragraphs should be indented by 1.27 cm.



\section{Conclus�o}

Section titles must be in boldface, 13pt, flush left. There should be an extra
12 pt of space before each title. Section numbering is optional. The first
paragraph of each section should not be indented, while the first lines of
subsequent paragraphs should be indented by 1.27 cm.



\bibliographystyle{sbc}
\bibliography{main}

\end{document}
